\documentclass[10pt]{article}
\textwidth 6.5in
\oddsidemargin=0in
\evensidemargin=0in

\usepackage{graphicx,bm,amssymb,amsmath,amsthm}

% general macros...
\newcommand{\bi}{\begin{itemize}}
\newcommand{\ei}{\end{itemize}}
\newcommand{\ben}{\begin{enumerate}}
\newcommand{\een}{\end{enumerate}}
\newcommand{\be}{\begin{equation}}
\newcommand{\ee}{\end{equation}}
\newcommand{\bea}{\begin{eqnarray}} 
\newcommand{\eea}{\end{eqnarray}}
\newcommand{\ba}{\begin{align}} 
\newcommand{\ea}{\end{align}}
\newcommand{\bse}{\begin{subequations}} 
\newcommand{\ese}{\end{subequations}}
\newcommand{\bc}{\begin{center}}
\newcommand{\ec}{\end{center}}
\newcommand{\bfi}{\begin{figure}}
\newcommand{\efi}{\end{figure}}
\newcommand{\ca}[2]{\caption{#1 \label{#2}}}
\newcommand{\ig}[2]{\includegraphics[#1]{#2}}
\newcommand{\tbox}[1]{{\mbox{\tiny #1}}}
\newcommand{\mbf}[1]{{\mathbf #1}}
\newcommand{\half}{\mbox{\small $\frac{1}{2}$}}
\newcommand{\vt}[2]{\left[\begin{array}{r}#1\\#2\end{array}\right]} % 2-col-vec
\newcommand{\mt}[4]{\left[\begin{array}{rr}#1&#2\\#3&#4\end{array}\right]} % 2x2
\newcommand{\RR}{\mathbb{R}}
\newcommand{\ZZ}{\mathbb{Z}}
\newcommand{\eps}{\varepsilon}
\newcommand{\bigO}{{\mathcal O}}
\newcommand{\intR}{\int_{-\infty}^\infty}
\DeclareMathOperator{\re}{Re}
\DeclareMathOperator{\im}{Im}
\newtheorem{thm}{Theorem}
\newtheorem{cnj}[thm]{Conjecture}
\newtheorem{lem}[thm]{Lemma}
\newtheorem{cor}[thm]{Corollary}
\newtheorem{pro}[thm]{Proposition}
\newtheorem{rmk}[thm]{Remark}
% this work...
\newcommand{\xx}{\mbf{x}}
\newcommand{\sss}{\mbf{s}}
\newcommand{\yy}{\mbf{y}}
\newcommand{\kk}{\mbf{k}}
\newcommand{\KK}{{\mathcal K}}
\newcommand{\NU}{{non-uniform}}
\newcommand{\U}{{uniform}}
\newcommand{\KB}{Kaiser--Bessel}
\newcommand{\wid}{\beta}               % exp sqrt width param
\newcommand{\sig}{\sigma}     % upsampling R ratio


\begin{document}
\title{FINUFFT: A lightweight non-uniform fast Fourier transform library}
\author{Alex Barnett and Jeremy Magland}
\date{\today}
\maketitle
\begin{abstract}
  Computation of Fourier transforms from data lying at arbitrary
  off-grid locations has many applications ranging from medical
  imaging to astronomy to fast algorithms for PDEs.  We present a
  software library for computing the non-uniform FFT (NFFT or NUFFT),
  of types 1 (\NU\ to \U), 2 (\U| to \NU), and
  3 (\NU\ to \NU), each in dimensions 1, 2, and 3.  Its
  main features are: a simple new kernel allowing faster on-the-fly
  spreading and interpolation from a regular grid, the use of
  quadrature rather than an analytic formula for the kernel Fourier
  transform, multi-threading for efficient use of shared-memory
  machines, simple calling interfaces matching those of the CMCL
  NUFFT library, and bindings to MATLAB/octave and python.  We
  compare performance of the library to existing CPU-based libraries; typically
  we match the runtime of the NFFT library of Potts et al.\ but
  without the need for a precomputation phase.
\end{abstract}




\section{Introduction}

The computational task addressed
is to evaluate the following exponential sums to a requested precision,
in optimal time.
The type-1 NUFFT (also known as the adjoint NFFT \cite{usingnfft})
in dimension $d$ %=1,2$ or 3
evaluates the Fourier series expansion for a set of
$M$ point sources at arbitrary locations $\xx_j$, which by periodicity
may be taken to lie in $[-\pi,\pi)^d$, and with
strengths $f_j\in\mathbb{C}$,  $j=1,\dots,M$.
The outputs are the Fourier modes with integer indices lying in
the set
\be
\KK = \KK_{N_1,\dots,N_d} := \KK_{N_1} \KK_{N_2} \dots \KK_{N_d}~,
\qquad\mbox{ where } \quad
\KK_{N_i} := \left\{\begin{array}{ll} \{-N_i/2,\ldots,N_i/2-1\}, & N_i \mbox{ even},\\
\{-(N_i-1)/2,\ldots,(N_i-1)/2\}, & N_i \mbox{ odd}.
\end{array}\right.
%\left[-\frac{N_1}{2},\frac{N_1-1}{2}\right] \times \dots \times \left[-\frac{N_d}{2},\frac{N_d-1}{2}\right]
\label{KK}
\ee
In the 1D case $\KK$ is an interval containing $N_1$ integer indices, in 2D it is a rectangle of $N_1N_2$ index pairs, and in 3D it is a cuboid of $N_1N_2N_3$ index triplets.
%\footnote{Note that these counts apply whether each $N_i$ is even or odd.
%  For instance for $N_1=10$ in 1D, the set is $\{-5,-4,\dots,4\}$
%  whereas for $N_1=11$, the set is $\{-5,-4,\dots,5\}$.
%}
We use $N=N_1\dots N_d$ to denote the total number of output values.
We do not address $d>3$ here.
Following the normalization in \cite{dutt,nufft}, then the type-1 NUFFT
evaluates
\be
F(\kk) := \frac{1}{M} \sum_{j=1}^M f_j e^{i \kk\cdot \xx_j}
\qquad \mbox{for } \kk \in \KK_{N_1,\dots,N_d}
\qquad \mbox{(Type-1, or \NU\ to \U\ transform)}
~.
\label{1}
\ee
The naive evaluation of $F$ at all indices $\kk$ requires $\bigO(NM)$
exponential evaluations, which is prohibitive
in many applications.
However, it is well known that by resampling onto a regular
grid and using the fast Fourier transform (FFT),
the work can be reduced to $\bigO(M |\log\eps|^d + N \log N)$,
where $\eps$ is the requested precision.
Here the term $|\log\eps|^d$ is the number of points on the regular
grid that a single source affects through the spreading kernel
of width $w=\bigO(|\log\eps|)$.
Regardless of $M$, the regular grid can be chosen a constant factor
$\sigma$ larger in each dimension than the numbers of modes needed $N_i$.
GIVE REF.
Our library also exploits this idea.

The type-2 transform (or NFFT \cite{usingnfft})
is, up to a normalization factor, the adjoint of the
type-1, and evaluates the Fourier series with given coefficients
$F(\kk)$, $\kk\in\KK$, at an arbitrary set of target points
$\xx_j$, $j=1,\ldots,M$, which due to periodicity may be taken to be in $[-\pi,\pi)^d$.
  That is,
  \be
  f_j := \sum_{\kk\in\KK_{N_1,\dots,N_d}} F(\kk) e^{i \kk\cdot \xx_j},
  \qquad j=1,\dots, M
\qquad \mbox{(Type-2, or \U\ to \NU\ transform)}
~.
\label{2}
\ee
Finally, the type-3 transform
\cite{nufft3} (or NNFFT \cite{usingnfft})
may be interpreted as evaluating the
Fourier transform of a set of sources at arbitrary locations $\xx_j$
in $\RR^d$
with strengths $f_j$, $j=1,\dots, M$, at the arbitrary target frequencies
$\sss_k$ in $\RR^d$, $k=1,\dots, N$. Note that here $k$ is a plain integer
index.
That is,
\be
F_k := \sum_{j=1}^M f_j e^{i \sss_k \cdot \xx_j}
  \qquad k=1,\dots, N
\qquad \mbox{(Type-3, or \NU\ to \NU\ transform)}
~.
\label{3}
\ee
Note that all three types of transform \eqref{1}, \eqref{2} and \eqref{3}
consist simply of computing an exponential sum.
In certain settings these may be interpretated as quadrature formulae
applied to Fourier transforms.
However, this is not to be confused with the ``inverse NUFFT'' which involves,
for instance, treating \eqref{2} as a large linear system to be solved for
$\{F(\kk)\}$ given $\{f_j\}$. The latter is common in Fourier imaging
applications; a popular solution method is to use
the preconditioned normal equations with
NUFFTs implementing the matrix-vector multiplies \cite{fessler,fourmont,fastsinc,gelbrecon}.







Applications.

MRI - non-Cartesian $k$-space trajectories.
CT.
Spectral interpolation from off-grid data.

Quadrature approximation of Fourier transforms, e.g.\
in image reconstruction \cite{cryo}.
Computation of spatially-periodic solutions to elliptic
PDEs via Ewald summation
\cite{lindbo11}.
Computation of history-dependent part for boundary-integral solvers
for the heat equation.%\cite{strain}.

Fast evaluation of random plane waves cite Beliaev.


Outline of algorithm:
EG type 1.
spreading to U grid using a kernel $\phi(\xx)$.
FFT.
Final correction of the Fourier modes by dividing by
$\hat\phi(k)$.


Existing implementations.

The first rigorous estimates of the type-1 and type-2 transforms
using the truncated Gaussian kernel was given in \cite{dutt}.

The CMCL NUFFT package \cite{cmcl} uses truncated Gaussians
and ``fast Gaussian gridding'' \cite[Sec.~3]{nufft}
which reduces the number of exponential evaluations
from $w^d$, where $w$ is the kernel width in grid-points,
to $(1+d)w$, giving a claimed 5--10 times acceleration.
On modern architectures RAM access is more of a bottleneck than
flops - DISCUSS.

Cite Nikos+Xiaobai.

The NFFT package \cite{nfft} from Chemnitz
includes the \KB\ kernel,
which
achieves roughly twice the number of digits of precision
compare to the truncated Gaussian, for the same kernel width.

Shared-memory parallelization of NFFT \cite{volkmer}.
Automatic tuning based on assumed Fourier coefficient decay \cite{nestler}.
A GPU implementation of the type-2 NUFFT has shown acceleration by around
a factor of 30 relative to NFFT in 1D and 2D \cite{cunfft}.

Fessler and Sutton \cite{fessler} use optimization in the space of
interpolation weights, enabling a slight increase of around 1/3 of a
digit in accuracy over \KB\ in 1D. However, they conclude that
``the
Kaiser--Bessel interpolator, with suitably optimized parameters,
represents a very reasonable compromise between accuracy and
simplicity.''



\bfi[t]  % fffffffffffffffffffffffffffffffffffffff
\ig{width=6.5in}{kernel.eps}
\ca{The new spreading kernel $\phi(z)$ given by
  \eqref{ES} and its Fourier transform.
  (a) shows the kernel for $\beta=4$; the discontinuities at
  $\pm 1$ are highlighted by dots.
  (b) shows a logarithmic plot of the (positive half of the)
  kernel for $\beta=30$
  (corresponding to a spreading width of 13 grid-points).
  The graph is a quarter-circle.
  (c) is a logarithmic plot of the magnitude of the
  Fourier transform of the kernel,
  showing the semicircular nature of the graph, and exponentially
  small values uniformly for $|\xi|>\beta$.
  This illustrates that ``the Fourier transform of the exponential
  of a semicircle is exponentially close to the exponential of a semicircle.''
}{f:kernel}
\efi
  

Here we present a kernel with performance essentially
equal to \KB, but which is simpler and faster to evaluate numerically.
In 1D, in rescaled spatial units, this kernel is
\be
\phi(z) = \phi_\beta(z) :=
\left\{\begin{array}{ll}
e^{\beta (\sqrt{1-z^2}-1)}, & |z|\le 1\\
0, & \mbox{otherwise}
\end{array}
\right.
\label{ES}
\ee
where $\wid>0$ is a width parameter that must be set
in accordance with the kernel width measured in uniform grid point units.
We call this the ``exponential square-root'' (ES) kernel;
see Fig.~\ref{f:kernel}.
In higher dimensions we use products of this 1D kernel.




The features of our implementation include:
\bi
\item use of the ES kernel with accuracy very similar to \KB\ but
  faster evaluation.
\item use of quadrature rather than an analytic formula to evaluate
  the kernel Fourier transform needed for the roll-off correction
  (deconvolution) phase.
\item parallel spreading and deconvolution on shared-memory machines via OpenMP.
\item use of the multi-threaded FFTW3 library for FFTs.
\item compilation options such as single-precision (to reduce RAM footprint)
  and/or single-threaded.
\item bindings to MATLAB, octave, and python.
\ei

%Unlike the authors of some other libraries,
We take the philosophy that the user calls the library to approximate the
exponential sums \eqref{1}--\eqref{3} to the requested precision.
Thus, the user does not have direct control over the type and width of
the spreading kernel; indeed, it would be confusing to have such control.
Rather, once the requested precision is given, such decisions
are made ``under the hood'' by the library.





\section{Usage of the library}

The interface to the library is very straightforward,
and 


\begin{verbatim}
finufft1d1(M,x,c,isign,acc,N,F,opts);
\end{verbatim}

where {\tt isign} is $\pm1$ and determines the sign of the imaginary
unit in \eqref{1}--\eqref{3}.

ETC



\section{Algorithms}

In our implementation we follow well-known procedures 
for evaluating the type-1, 2 and 3 NUFFTs.
We describe them here for convenience.
We use the Fourier transform convention
\be
\hat\phi(k) = \frac{1}{2\pi} \intR \phi(x) e^{-ikx} dx
~,\qquad
\phi(x) = \intR \hat\phi(k) e^{ikx} dx
~.
\label{ft}
\ee
We will need a
phased version of the Poisson summation formula REF which states that
for any function
$\psi \in L_1(\RR)$ with Fourier transform $\hat\psi$ is then
\be
\sum_{l\in\ZZ} e^{il\theta} \psi(x - l\delta) \; = \;
\frac{2\pi}{\delta} \sum_{m\in\ZZ}
\hat\psi\biggl(\frac{2\pi m + \theta}{\delta}\biggr)
e^{i(2\pi m + \theta)x /\delta}
\label{pois}
\ee
which can be proven as usual by writing the Euler--Fourier
formula for the Fourier series coefficients
of the left-hand side of \eqref{pois} multiplied by $e^{-ix\theta/\delta}$,
wrich turns it into a periodic function.



\subsection{Type 1}

Firstly, given a requested precision $\eps$, an integer kernel width $w$
is chosen, such that $w=\bigO(|log \eps|$). The precise choice of
$w$ is given later ***.

An upsampling ratio $\sig>1$ is chosen,
so that the DFT size $n = \sig N$, where $N$ is the number of modes.
Following many researchers, we find $\sig=2$ adequate.
In fact, for efficiency of the FFT, and convenience in the spreading
code, we then set $n$ to be the smallest
integer of the form $2^q3^p5^r$ greater than or equal to the
larger of $\sig N$ and $2w$.





\subsection{Type 2}


\subsection{Type 3}

Can break it by choosing $XK$ huge, even with 2 input and 2 output pts.


\section{Error analysis}

FT defns.

\subsection{Error in the type-2 transform with general kernel}

see \cite[Sec.~V.B]{fessler}.




\subsection{Esimates on the Fourier transform of the ES kernel}

Discuss relation of kernel \eqref{ES} to \KB.





scaled to width.

$w=1,2,\ldots$
sets the support of the kernel in units of the uniform grid point spacing, and





\section{Conclusion}



GPU version; however tricky for spreading to \U\ grid, needed in types 1 and 3.


Remains open to prove a relation between the zero prolate spheroidal
wavefunction and the ES kernel \eqref{ES}.




% BBBBBBBBBBBBBBBBBBBBBBBBBBBBBBBBBBBBBBBBBBBBBBBBBBBBBBBBBBBBBBBBBBBBBBBBBBBB
\bibliographystyle{abbrv}
\bibliography{alex}
\end{document}
